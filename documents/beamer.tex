\documentclass[english, aspectratio=169]{beamer}
% english is for the language used in standard texts (figures, tables etc)
% aspectratio of 16:9 or set it for more old school to 4:3 (without the ':')

% ---------------------------------------------------------------------------- %
% Load base preamble
% ---------------------------------------------------------------------------- %

% Import from absolute path
% \usepackage{import}
% \import{C:/GitHub/LaTeX_Preamble_and_Examples/preamble/}{beamer_en.tex}

% Import from a relative path
\usepackage{import}
\subimport{../preamble/}{beamer_en.tex}

% ---------------------------------------------------------------------------- %
% Local macros
% ---------------------------------------------------------------------------- %

% It is quite common that you are going to write the same thing multiple times.
% As part of the preamble I have predefined some macros for what I keep on
% having to use, and while I endorse you send a pull request with more, here you
% can add the ones you only need for this one document.

% ---------------------------------------------------------------------------- %
% Bibliography
% ---------------------------------------------------------------------------- %
\usepackage[
backend=biber,
style=alphabetic,
sortlocale=EN_GB,
useprefix=false,
url=false,
doi=true,
eprint=false,
maxbibnames=99,
]{biblatex}
\addbibresource{references.bib}


% ----------------------------------------------------------------------------------------
% TITLEPAGE
% ----------------------------------------------------------------------------------------
\title[Alternate title]{
  {\small Reverse engineerable \LaTeX examples:} \\ {Beamer presentations}
}

\author{Steffan S\o lvsten}
\institute[AU] {
  Aarhus University
}


% ---------------------------------------------------------------------------- %
% Content
% ---------------------------------------------------------------------------- %
\begin{document}

\begin{frame}
  \titlepage
\end{frame}

\begin{frame}
  \tableofcontents
\end{frame}

\section{Basic slides}
\begin{frame}
  This is a very simple slide, which contains some math symbols such as
  $\sigma$, and also a math equation such as
  \begin{equation*}
    \sum_{i=1}^n i = \frac{n(n+1)}{2}
  \end{equation*}
  Here is a theorem with a proof

  \begin{theorem}[Euler]
    If $n$ and $a$ are coprime positive integers, then $a^{\phi(n)} \equiv 1
    \mod n$
  \end{theorem}
  \begin{proof}
    Left to the reader.
  \end{proof}
\end{frame}

\begin{frame}
  It is quite common to have your slide divided into two or more parts. This is
  done using columns.

  \begin{columns}
    \begin{column}{0.5\textwidth}
      Here is some text in the left column
    \end{column}
    \begin{column}{0.5\textwidth}
      Here is some text in the right column
    \end{column}
  \end{columns}
\end{frame}

\section{Code}
\begin{frame}[fragile]
  You can have code snippets on your slides just the same way you do in the
  documents. You only have to mark the frame \emph{fragile}, such that overflows
  do not immediately break the compilation.

  \begin{lstlisting}[language=coq, caption={Exercise on proving strong induction
      in \cite{SFDiscrete}}]
  Theorem strong_induction :
    forall P : nat -> Prop,
    (forall n : nat, (forall m : nat, m < n -> P m) -> P n) ->
    forall n : nat, P n.
  Proof.
    intros P IH_strong n.
    assert (H : forall k, k <= n -> P k).
    { ... }
    now apply H.
  Qed.
  \end{lstlisting}
\end{frame}

\section{Figures, Tables, and Graphics}
\begin{frame}
  \frametitle{Tables}
  Tables are just as simple as you are used to.

  \begin{table}
    \centering
    \begin{tabular}{c|c}
      A cell & Another one
      \\ \hline
      $\gamma$ & $\beta$
    \end{tabular}
    
    \caption{A table}
    \label{tab:label}
  \end{table}  
\end{frame}


\begin{frame}
  \frametitle{Graphics}
  Figures are also exactly as easy. That means, you can actually have the whole
  figure in another file, that you can include in both an article and in a
  presentation!

  \begin{figure}
    \centering
    \begin{tikzpicture}[shorten >= 1, node distance = 2cm, on grid, minimum size = 0.8cm]
  % Threats
  \node[shape=circle, draw=black, label=above:$6$, label=left:$\rightarrow$] (threat1) {};
  \node[draw=none] (threat1_t) [below=1.3cm of threat1] {$(0,0,0,0,0,\frac{1}{2 n^2},0)$};
  \node[shape=circle, draw=black, label=above:$7$] (threat2) [right=3cm of threat1] {};
  \node[draw=none] (threat2_t) [below=1.3cm of threat2] {$(0,0,0,0,0,0,\frac{1}{2n^2})$};

  % Chance node
  \node[shape=diamond, draw=black] (chance_node) [right=2cm of threat2] {};

  % Subgames
  \node[draw=none] (g1_1) [above right=1.2cm and 3cm of chance_node] {$\Gmul(x_1,x_1,\frac{1 + a_{1,1}^k}{2})$};
  \node[draw=none] (gi_j) [above right=0cm and 3cm of chance_node] {$\Gmul(x_i,x_j,\frac{1 + a_{i,j}^k}{2})$};
  \node[draw=none] (gn_n) [below right=1.2cm and 3cm of chance_node] {$\Gmul(x_n,x_n,\frac{1 + a_{n,n}^k}{2})$};
  
  % HACK: Nodes for dashed arrows
  \node[draw=none] (gi) [below right=0.6cm and 2cm of chance_node] {};
  \node[draw=none] (gj) [above right=0.6cm and 2cm of chance_node] {};
  % Edges
  \path[->]
    (threat1) edge (threat2)
    (threat1) edge (threat1_t)
    (threat2) edge (threat2_t)
    (threat2) edge (chance_node)
    (chance_node) edge [bend left=20] node [left] {$\frac{1}{n^2}$} (g1_1)
    (chance_node) edge (gi_j)
    (chance_node) edge [bend right=25] node [left] {$\frac{1}{n^2}$} (gn_n)
  ;
  \path[->]
    (chance_node) edge [bend right=10] (gi)
    (chance_node) edge [bend left=10] (gj)[loosely dashed]
  ;
  % Dots
  \path[-]
%      (g1_1) edge [bend right=10] (gi_j)
%      (gi_j) edge [bend right=10] (gn_n) [loosely dotted, thick]
    (g1_1) edge (gi_j)
    (gi_j) edge (gn_n) [loosely dotted, thick]
  ;
\end{tikzpicture}
  
%%% Local Variables:
%%% mode: latex
%%% TeX-master: "../examples.tex"
%%% End:


    \caption{The label for the figure.}
    \label{fig:label}
  \end{figure}
\end{frame}


\section{Animations}
\subsection{Simple use of pause and onslide}
\begin{frame}
  \frametitle{pause command}
  To animate slides it is luckily not necessary to copy paste all of the code
  and change one thing after the other! Otherwise Bærbak would definitely have
  been very sad.
  
  The simplest animations you need is slowly revealing the slide, such as the
  bullet points below. This is done using the \lstinline{\\pause}
  command.

  \begin{itemize}
  \item An item, that is shown at the very beginning.

    \pause
    
  \item An item, that is first shown on the ``next slide''
  \end{itemize}

  \pause
  I also at times get a weird \lstinline{TeX capacity exceeded} error when using
  \lstinline{enumerate} instead of \lstinline{itemize}. So, it is better to just
  stick to that one.

  Finally, if you need to compile handouts without animations, just add
  \lstinline{handout} in the \lstinline{documentclass} above
\end{frame}

\begin{frame}
  \frametitle{onslide command}

  To do more complicated animations you want to trigger parts of the slide at
  different times. You do this using the \lstinline{\\onslide<a-b>} where $a$,
  and $b$ are optional slide numbers.

  \onslide<2->{This paragraph is shown on the second step and forwards}

  \onslide<2-3>{This paragraph is shown on the second and third steps of the slide}

  \onslide<1-2>{This paragraph is shown on the first two steps of the slides}

  \onslide<4>{This paragraph is shown on the fourth (last) step of the slides}
\end{frame}

\subsection{Tikz animations}
\begin{frame}
  \frametitle{Animated Tikz}

  \begin{columns}
    \begin{column}{0.5\linewidth}
      Here is a simple example using the \lstinline{\\onslide} command to
      trigger different parts of the slide at different times.
    \end{column}
    \begin{column}{0.5\linewidth}
      \begin{tikzpicture}[thick]
        \node[state, label=left:$\rightarrow$] (q0) {$q_0$};
        \node[state, above left=of q0] (q1) {$q_1$};
        \node[state, below left=1cm and 0.2cm of q0] (q2) {$q_2$};
        \node[state, below right=1cm and 0.2cm of q0] (q3) {$q_3$};
        \node[state, accepting, above right=of q0] (q4) {$q_4$};

        % path 1
        \draw[->]
        (q0) edge[bend left] node[below left] {$a$} (q1)
        (q1) edge[bend left] node[above right] {$a$} (q0)
        (q0) edge node[below right] {$b$} (q4)
        ;

        % path 1
        \draw[->]
        (q0) edge[bend right] node[left] {$a$} (q2)
        (q2) edge[bend right] node[below] {$a$} (q3)
        (q3) edge[bend right] node[right] {$b$} (q0)
        ;

        \onslide<2->{
          \draw[->, cyan]
          (q0) edge[bend left] (q1)
          (q1) edge[bend left] (q0)
          (q0) edge (q4)
          ;
          \node[state, accepting, above right=of q0, cyan] (q4) {$q_4$};
        }
      \end{tikzpicture}
    \end{column}
  \end{columns}
\end{frame}

\begin{frame}
  \frametitle{Animated Tikz}

  \begin{columns}
    \begin{column}{0.5\linewidth}
      Here is a much more complicated example together with an animation in
      sync below.

      \begin{align*}
        \{ q_0 \} &\onslide<2-5>{\rightarrow \{ q_1, q_2 \}}
        \\ &\onslide<3-5>{\rightarrow \{ q_0, q_3 \}}
        \\ &\onslide<4-5>{\rightarrow \{ q_0, q_4 \}}
      \end{align*}

      \onslide<4-5>{Are any of the final states accepting?}
      \onslide<5-5>{Yes, $q_4$ is!}
    \end{column}
    \begin{column}{0.5\linewidth}
      \begin{tikzpicture}[thick]
        \node[state, label=left:$\rightarrow$] (q0) {$q_0$};
        \node[state, above left=of q0] (q1) {$q_1$};
        \node[state, below left=1cm and 0.2cm of q0] (q2) {$q_2$};
        \node[state, below right=1cm and 0.2cm of q0] (q3) {$q_3$};
        \node[state, accepting, above right=of q0] (q4) {$q_4$};

        % path 1
        \draw[->]
          (q0) edge[bend left] node[below left] {$a$} (q1)
          (q1) edge[bend left] node[above right] {$a$} (q0)
          (q0) edge node[below right] {$b$} (q4)
        ;

        % path 1
        \draw[->]
          (q0) edge[bend right] node[left] {$a$} (q2)
          (q2) edge[bend right] node[below] {$a$} (q3)
          (q3) edge[bend right] node[right] {$b$} (q0)
        ;

        % animation
        \onslide<1-1>{
          \node[state, label=left:$\rightarrow$, cyan] {$q_0$};
        }
  
        \onslide<2-2>{
          \draw[->, cyan]
            (q0) edge[bend left] node[below left] {$a$} (q1)
            (q0) edge[bend right] node[left] {$a$} (q2)
          ;
          \node[state, above left=of q0, cyan] {$q_1$};
          \node[state, below left=1cm and 0.2cm of q0, cyan] {$q_2$};
        }
      
        \onslide<3-3>{
          \draw[->, cyan]
            (q1) edge[bend left] node[above right] {$a$} (q0)
            (q2) edge[bend right] node[below] {$a$} (q3)
          ;
          \node[state, label=left:$\rightarrow$, cyan] {$q_0$};
          \node[state, below right=1cm and 0.2cm of q0, cyan] {$q_3$};
        }
      
        \onslide<4-4>{
          \draw[->, cyan]
            (q0) edge node[below right] {$b$} (q4)
            (q3) edge[bend right] node[right] {$b$} (q0)
          ;
          \onslide<4-5>
          \node[state, cyan] {$q_0$};
          \node[state, accepting, above right=of q0, cyan] {$q_4$};
        }
      \end{tikzpicture}
    \end{column}
  \end{columns}  
\end{frame}

\subsection{Animating code}
\begin{frame}[fragile]
  By escaping from lstlisting out to \LaTeX\ using the
  \texttt{*@}\dots\texttt{@*} macro you can also slowly reveal code. This could
  be useful for showing a \emph{Coq} proof as below
  \begin{lstlisting}[language=coq]
  Theorem plus_n_O : forall n:nat, n = n + 0.
  Proof.*@\pause@* 
    induction n as [| n' IHn'].*@\pause@*
    - (* n = 0 *)
      reflexivity.*@\pause@*
    - (* n = S n' *)
      simpl. rewrite <- IHn'. reflexivity.*@\pause@*
  Qed.
  \end{lstlisting}
\end{frame}

\printbibliography
\end{document}
