% ---------------------------------------------------------------------------- %
% Simple macros
% ---------------------------------------------------------------------------- %
\newcommand{\Id}{\text{Id}}                                                      % t\Id = Identity function

\newcommand{\Det}[1]{\text{Det}\left( #1 \right)}                                % \Det{arg}             Det(arg)
\newcommand{\Span}[1]{\text{Span}\left( #1 \right)}                              % \Span{arg}            Span(arg)
\newcommand{\sgn}[1]{\text{sgn} \left( #1 \right)}                               % \sgn{arg}             sgn(arg)
\newcommand{\adj}[1]{\text{adj} \left( #1 \right)}                               % \adj{arg}             adj(arg)
\newcommand{\ord}[1]{\text{ord} \left( #1 \right)}                               % \ord{arg}             ord(arg)

\newcommand{\tuple}[1]{\left\langle #1 \right\rangle}                            % \tuple{arg}           <arg>
\newcommand{\abs}[1]{\left\lvert #1 \right\rvert}                                % \abs{arg}             absolute/modulo of value
\newcommand{\norm}[1]{\left\lVert #1 \right\rVert}                               % \norm{arg}            norm of a value
\newcommand{\ceil}[1]{\left\lceil #1 \right\rceil}                               % \ceil{arg}            ceiling of a value
\newcommand{\floor}[1]{\left\lfloor #1 \right\rfloor}                            % \floor{arg}           floor of a value
\newcommand{\inprod}[2]{\tuple{#1, #2}}                                          % \inprod{v}{w}         inner product
\newcommand{\powgroup}[1]{\tuple{#1}}                                            % \powgroup{arg}        image of f={g^n | n in Z}

% ---------------------------------------------------------------------------- %
% Advanced macros:
% ---------------------------------------------------------------------------- %
\usepackage{xparse}                                                              % Scanning arguments
\usepackage{xifthen}                                                             % Conditionals
\usepackage{xstring}                                                             % String functions
\usepackage{calc}                                                                % Calculations

\newcounter{i}

\DeclareDocumentCommand \set { m g g }{                                          % \set{X}{C}{|}
     \left\lbrace                                                                % {X | C}
         #1 \IfValueT {#2} { \ \IfValueTF{#3}{#3}{|}\  #2 }
     \right\rbrace
}

\DeclareDocumentCommand \seq { g g g g } {                                       % \seq{x}{i}{j}{s}
    \setcounter{i}{0}                                                            % x_i, x_i+s, ... x_j
    \IfValueT {#2} { \addtocounter{i}{#2} }
    \IfValueTF {#1}
        {#1}
        {x}
    _{ \arabic{i} },
    \IfValueTF {#4} 
        {\addtocounter{i}{#4}}
        {\addtocounter{i}{1}}
    \IfValueTF {#1} 
        {#1}
        {x} 
    _{ \arabic{i} },
    \dots
    \IfValueTF {#3}
        { , #1_{#3} }
        {}
}

\DeclareDocumentCommand \eqclass { g g }{                                        % \eqclass{v}{V}    Equivalent Class
    \left[                                                                       % This can also be used for coordinate vectors
        \IfValueTF{#1}
            {#1}
            {\dot}
    \right]
    \IfValueTF{#2}
            {_{#2}}
            {}
}

\DeclareDocumentCommand \ero { g g } {                                           % \ero {x, y}
    \begin{array}{c}                                                             %    x
        \IfValueTF{#1}                                                           %    ~
            {_{#1}}                                                              %    y
            {\phantom{\sim}}
    \\
        \sim
    \\
        \IfValueTF{#2}
            {^{#2}}
            {\phantom{\sim}}
    \end{array}
}

\DeclareDocumentCommand \matrep { g g g } {                                      % \matrep{W}{L}{V}    Matrixrepresentation
    {_{                                                                          % W[L]V
        \IfValueTF {#1}                                                          %   No arguments for W or V results in standard basis
            {#1}                                                                 %   No arguments for L results in the coordinate transformation
            {\epsilon}
    }}
    \left[
        \IfValueTF {#2}
            {#2}
            {\square}
    \right] {_{
        \IfValueTF {#3}
            {#3}
            {\epsilon}
    }}
}

\newcommand{\IndexedFunc}[3]{{#1}_{#2} \left( #3 \right)}

\DeclareDocumentCommand \Geo { g g }{                                            % \Geo{v}{V}    Geogrebic Multiplicity
    \IndexedFunc
        {\text{Geo}}
        {\IfValueTF{#1}
                {#1}
                {L}}
        {\IfValueTF{#2}
            {#2}
            {\lambda}}
}

\DeclareDocumentCommand \Alg { g g }{                                            % \Alg            Algebraic Multiplicity
    \IndexedFunc
        {\text{Alg}}
        {\IfValueTF{#1}
                {#1}
                {L}}
        {\IfValueTF{#2}
            {#2}
            {\lambda}}
}

\DeclareDocumentCommand \series { g g g g }{                                     % \series{a}{b}{c}{d}    \{ a_b \} _{b = c} ^d
    \set{\IfValueTF{#1}                                                          %    An infinite series with elements a, indexed by b
      {#1}                                                                       %    starting from c ending at d
      {a}
     _{\IfValueTF{#2}
      {#2}
      {n}}      
      }
      _{\IfValueTF{#2}
        {#2}
        {n}
       =
       \IfValueTF{#3}
        {#3}
        {1}
      }
      ^{\IfValueTF{#4}
        {#4}
        {\infty}
      }
}

\DeclareDocumentCommand \pseries { g g g g }{                                    % \pseries{a}{b}{c}{d}   \{ x^k \} _{k = c} ^d
    \set{\IfValueTF{#1}                                                          %    An infinite series of points x, indexed by k
      {#1}                                                                       %    starting from c ending at d
      {x}
     ^{\IfValueTF{#2}
      {#2}
      {k}}      
      }
      _{\IfValueTF{#2}
        {#2}
        {k}
       =
       \IfValueTF{#3}
        {#3}
        {1}
      }
      ^{\IfValueTF{#4}
        {#4}
        {\infty}
      }
}

\DeclareDocumentCommand \infseq { g g g g }{                                     % \infseq{a}{b}{c}{d}   \sum_{ b = c }^d a_b
  \sum                                                                           %    An infinite sequence with elements a, indexed by b
    _{                                                                           %    starting from c ending at d
        \IfValueTF{#2}
          {#2}
          {n}
         =
         \IfValueTF{#3}
          {#3}
          {1}
    }
      ^{\IfValueTF{#4}
        {#4}
        {\infty}
      }
    \IfValueTF{#1}
      {#1}
      {a}
   _{\IfValueTF{#2}
      {#2}
      {n}
  }     
}

\renewcommand{\|}{\scalebox{1.3}{|}\ }                                           % Larger vertical bar for divisors
\renewcommand{\div}[1]{\text{div}\left( #1 \right)}                              % div(arg)
