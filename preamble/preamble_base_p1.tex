% ---------------------------------------------------------------------------- %
% Paper size and encoding
% ---------------------------------------------------------------------------- %

\usepackage[utf8]{inputenc}                                                      % encoding: utf-8 (nordic letters)
\usepackage[T1]{fontenc}                                                         % use 8-bit encoded fonts

% Uncomment to add more whitespace between paragraphs rather than indent
%\usepackage[parfill]{parskip}

\usepackage{csquotes}

% ---------------------------------------------------------------------------- %
% Tables and figures
% ---------------------------------------------------------------------------- % 
\usepackage{tabularx,booktabs,authblk}                                           % various basic stuff for tables and more
\usepackage{caption, subcaption}                                                 % captions and sub figures

\usepackage{wrapfig}                                                             % figures wrapped by text

\usepackage{rotating}                                                            % rotate content (i.e. figures sideways)

% ---------------------------------------------------------------------------- %
% Hypermedia
% ---------------------------------------------------------------------------- %

\usepackage{url}                                                                 % \url{link}
\usepackage[hidelinks]{hyperref}                                                 % \href{link}{replacing text}

\newcommand*{\http}[1]{\href{http://#1}{#1}}                                     % macro for http links: \http{www.matfystutor.dk}
\newcommand*{\mailto}[1]{\href{mailto:#1}{\nolinkurl{#1}}}                       % macro for mails: \mailto{email@email.com}

% ---------------------------------------------------------------------------- %
% Mathematics
% ---------------------------------------------------------------------------- %
% various basic stuff
\usepackage{mathtools, amsmath, amssymb}
\usepackage{stmaryrd}                                                            % even more symbols

% Pull requests with more macros are very welcome!

% ---------------------------------------------------------------------------- %
% Linear Algebra
% ---------------------------------------------------------------------------- %
\newcommand{\allones}{\ensuremath{\mathbf{1}}}                                  % \allones  = The all ones vector
\newcommand{\allzeros}{\ensuremath{\mathbf{0}}}                                 % \allzeros = The all zeros vector

% ---------------------------------------------------------------------------- %
% Calligraphic symbols ('cal' prefix)
% ---------------------------------------------------------------------------- %
\newcommand{\calF}{\ensuremath{\mathcal{F}}}
\newcommand{\calG}{\ensuremath{\mathcal{G}}}
\newcommand{\calH}{\ensuremath{\mathcal{H}}}
\newcommand{\calO}{\ensuremath{\mathcal{O}}}
\newcommand{\calS}{\ensuremath{\mathcal{S}}}
\newcommand{\calV}{\ensuremath{\mathcal{V}}}
\newcommand{\calW}{\ensuremath{\mathcal{W}}}

% ---------------------------------------------------------------------------- %
% Algebra
% ---------------------------------------------------------------------------- %
\renewcommand{\d}{\, \mathrm{d}}                                                 % \d = differential d with a bit of spacing
\newcommand{\e}{\ensuremath{\mathrm{e}}}                                         % \e = eulers number
\newcommand{\R}{\ensuremath{\mathbb{R}}}                                         % \R = Real numbers
\newcommand{\N}{\ensuremath{\mathbb{N}}}                                         % \N = Natural numbers
\newcommand{\Z}{\ensuremath{\mathbb{Z}}}                                         % \Z = Integers
\newcommand{\C}{\ensuremath{\mathbb{C}}}                                         % \C = Complex numbers
\newcommand{\Q}{\ensuremath{\mathbb{Q}}}                                         % \Q = Rational numbers
\newcommand{\F}{\ensuremath{\mathbb{F}}}                                         % \F = Field
\newcommand{\K}{\ensuremath{\mathbb{K}}}                                         % \K = Field \R and \C
\renewcommand{\S}{\ensuremath{\mathbb{S}}}                                       % \S = Group of permutations

% ---------------------------------------------------------------------------- %
% Complexity Theory ('c' prefix)
% ---------------------------------------------------------------------------- %
% Complexity classes
\newcommand{\cDSPACE}{\text{DSPACE}}                                             % \cDSPACE   = DSPACE
\newcommand{\cDTIME}{\text{DTIME}}                                               % \cDTIME    = DTIME
\newcommand{\cNSPACE}{\text{NSPACE}}                                             % \cNSPACE   = NSPACE
\newcommand{\cNTIME}{\text{NTIME}}                                               % \cNTIME    = NTIME

\newcommand{\cCLASS}[1]{\ensuremath{\mathrm{#1}}}

\newcommand{\cL}{\cCLASS{L}}                                                     % \cL        = Deterministic Logarithmic Space
\newcommand{\cNL}{\cCLASS{NL}}                                                   % \cNL       = Nondeterministic Logarithmic Space
\newcommand{\cP}{\cCLASS{P}}                                                     % \cP        = Deterministic Polynomial Time
\newcommand{\cNP}{\cCLASS{NP}}                                                   % \cNP       = Nondeterministic Polynomial Time
\newcommand{\ccoNP}{\cCLASS{coNP}}                                               % \ccoNP     = Conondeterministic Polynomial Time
\newcommand{\cPSPACE}{\cCLASS{PSPACE}}                                           % \cPSPACE   = Deterministic Polynomial Space
\newcommand{\cEXP}{\cCLASS{EXP}}                                                 % \cEXP      = Deterministic Polynomial Time
\newcommand{\cNEXP}{\cCLASS{NEXP}}                                               % \cNEXP     = Nondeterministic Polynomial Time
\newcommand{\cEXPSPACE}{\cCLASS{EXPSPACE}}                                       % \cEXPSPACE = Deterministic Polynomial Space

\newcommand{\cSqrtSum}{\cCLASS{SqrtSum}}                                          % \SqrtSum   = Square Root sum complexity class
\newcommand{\cETR}{\cCLASS{\exists\R}}                                           % \cETR      = Existential Theory of the Reals

\newcommand{\cBPP}{\cCLASS{BPP}}                                                 % \cBPP      = Polynomial time randomness with 2-sided error
\newcommand{\cFPT}{\cCLASS{FPT}}                                                 % \cFPT      = Fixed parameter tractable

% Complete problems
\newcommand{\cSTCON}{\textsc{STCON}}                                             % \cSTCON    = ST Connectivity
\newcommand{\cSAT}{\textsc{SAT}}                                                 % \cSAT      = Satisfiability
\newcommand{\cILP}{\textsc{ILP}}                                                 % \cILP      = Integer Linear Programming
\newcommand{\cOVP}{\textsc{OVP}}                                                 % \cOVP      = Orthogonal Vectors

% Hypothesis
\newcommand{\cETH}{\textsc{eth}}                                                 % \cETH      = Exponential Time Hypothesis
\newcommand{\cSETH}{\textsc{Seth}}                                               % \cSETH     = Strong ETH

% ---------------------------------------------------------------------------- %
% Simple macros
% ---------------------------------------------------------------------------- %
\newcommand{\Id}{\text{Id}}                                                      % t\Id = Identity function

\newcommand{\Det}[1]{\text{Det}\left( #1 \right)}                                % \Det{arg}             Det(arg)
\newcommand{\Span}[1]{\text{Span}\left( #1 \right)}                              % \Span{arg}            Span(arg)
\newcommand{\sgn}[1]{\text{sgn} \left( #1 \right)}                               % \sgn{arg}             sgn(arg)
\newcommand{\adj}[1]{\text{adj} \left( #1 \right)}                               % \adj{arg}             adj(arg)
\newcommand{\ord}[1]{\text{ord} \left( #1 \right)}                               % \ord{arg}             ord(arg)

\newcommand{\tuple}[1]{\left\langle #1 \right\rangle}                            % \tuple{arg}           <arg>
\newcommand{\abs}[1]{\left\lvert #1 \right\rvert}                                % \abs{arg}             absolute/modulo of value
\newcommand{\norm}[1]{\left\lVert #1 \right\rVert}                               % \norm{arg}            norm of a value
\newcommand{\ceil}[1]{\left\lceil #1 \right\rceil}                               % \ceil{arg}            ceiling of a value
\newcommand{\floor}[1]{\left\lfloor #1 \right\rfloor}                            % \floor{arg}           floor of a value
\newcommand{\inprod}[2]{\tuple{#1, #2}}                                          % \inprod{v}{w}         inner product
\newcommand{\powgroup}[1]{\tuple{#1}}                                            % \powgroup{arg}        image of f={g^n | n in Z}

% ---------------------------------------------------------------------------- %
% Advanced macros:
% ---------------------------------------------------------------------------- %
\usepackage{xparse}                                                              % Scanning arguments
\usepackage{xifthen}                                                             % Conditionals
\usepackage{xstring}                                                             % String functions
\usepackage{calc}                                                                % Calculations

\newcounter{i}

\DeclareDocumentCommand \set { m g g }{                                          % \set{X}{C}{|}
     \left\lbrace                                                                % {X | C}
         #1 \IfValueT {#2} { \ \IfValueTF{#3}{#3}{|}\  #2 }
     \right\rbrace
}

\DeclareDocumentCommand \seq { g g g g } {                                       % \seq{x}{i}{j}{s}
    \setcounter{i}{0}                                                            % x_i, x_i+s, ... x_j
    \IfValueT {#2} { \addtocounter{i}{#2} }
    \IfValueTF {#1}
        {#1}
        {x}
    _{ \arabic{i} },
    \IfValueTF {#4} 
        {\addtocounter{i}{#4}}
        {\addtocounter{i}{1}}
    \IfValueTF {#1} 
        {#1}
        {x} 
    _{ \arabic{i} },
    \dots
    \IfValueTF {#3}
        { , #1_{#3} }
        {}
}

\DeclareDocumentCommand \eqclass { g g }{                                        % \eqclass{v}{V}    Equivalent Class
    \left[                                                                       % This can also be used for coordinate vectors
        \IfValueTF{#1}
            {#1}
            {\dot}
    \right]
    \IfValueTF{#2}
            {_{#2}}
            {}
}

\DeclareDocumentCommand \ero { g g } {                                           % \ero {x, y}
    \begin{array}{c}                                                             %    x
        \IfValueTF{#1}                                                           %    ~
            {_{#1}}                                                              %    y
            {\phantom{\sim}}
    \\
        \sim
    \\
        \IfValueTF{#2}
            {^{#2}}
            {\phantom{\sim}}
    \end{array}
}

\DeclareDocumentCommand \matrep { g g g } {                                      % \matrep{W}{L}{V}    Matrixrepresentation
    {_{                                                                          % W[L]V
        \IfValueTF {#1}                                                          %   No arguments for W or V results in standard basis
            {#1}                                                                 %   No arguments for L results in the coordinate transformation
            {\epsilon}
    }}
    \left[
        \IfValueTF {#2}
            {#2}
            {\square}
    \right] {_{
        \IfValueTF {#3}
            {#3}
            {\epsilon}
    }}
}

\newcommand{\IndexedFunc}[3]{{#1}_{#2} \left( #3 \right)}

\DeclareDocumentCommand \Geo { g g }{                                            % \Geo{v}{V}    Geogrebic Multiplicity
    \IndexedFunc
        {\text{Geo}}
        {\IfValueTF{#1}
                {#1}
                {L}}
        {\IfValueTF{#2}
            {#2}
            {\lambda}}
}

\DeclareDocumentCommand \Alg { g g }{                                            % \Alg            Algebraic Multiplicity
    \IndexedFunc
        {\text{Alg}}
        {\IfValueTF{#1}
                {#1}
                {L}}
        {\IfValueTF{#2}
            {#2}
            {\lambda}}
}

\DeclareDocumentCommand \series { g g g g }{                                     % \series{a}{b}{c}{d}    \{ a_b \} _{b = c} ^d
    \set{\IfValueTF{#1}                                                          %    An infinite series with elements a, indexed by b
      {#1}                                                                       %    starting from c ending at d
      {a}
     _{\IfValueTF{#2}
      {#2}
      {n}}      
      }
      _{\IfValueTF{#2}
        {#2}
        {n}
       =
       \IfValueTF{#3}
        {#3}
        {1}
      }
      ^{\IfValueTF{#4}
        {#4}
        {\infty}
      }
}

\DeclareDocumentCommand \pseries { g g g g }{                                    % \pseries{a}{b}{c}{d}   \{ x^k \} _{k = c} ^d
    \set{\IfValueTF{#1}                                                          %    An infinite series of points x, indexed by k
      {#1}                                                                       %    starting from c ending at d
      {x}
     ^{\IfValueTF{#2}
      {#2}
      {k}}      
      }
      _{\IfValueTF{#2}
        {#2}
        {k}
       =
       \IfValueTF{#3}
        {#3}
        {1}
      }
      ^{\IfValueTF{#4}
        {#4}
        {\infty}
      }
}

\DeclareDocumentCommand \infseq { g g g g }{                                     % \infseq{a}{b}{c}{d}   \sum_{ b = c }^d a_b
  \sum                                                                           %    An infinite sequence with elements a, indexed by b
    _{                                                                           %    starting from c ending at d
        \IfValueTF{#2}
          {#2}
          {n}
         =
         \IfValueTF{#3}
          {#3}
          {1}
    }
      ^{\IfValueTF{#4}
        {#4}
        {\infty}
      }
    \IfValueTF{#1}
      {#1}
      {a}
   _{\IfValueTF{#2}
      {#2}
      {n}
  }     
}

\renewcommand{\|}{\scalebox{1.3}{|}\ }                                           % Larger vertical bar for divisors
\renewcommand{\div}[1]{\text{div}\left( #1 \right)}                              % div(arg)


% adds vertical lines to matrices
\makeatletter
\renewcommand*\env@matrix[1][*\c@MaxMatrixCols c]{
  \hskip -\arraycolsep
  \let\@ifnextchar\new@ifnextchar
  \array{#1}}
\makeatother


% ---------------------------------------------------------------------------- %
% Logic and proofs
% ---------------------------------------------------------------------------- %
% Proofs
\usepackage{amsthm}                                                              % Theorem package
\theoremstyle{definition}                                                        % Style: plain, definition, remark

% Uncomment if you want to have the number first
%\swapnumbers

% Logic packages
\usepackage{lplfitch}                                                            % fitch style proofs
\usepackage{bussproofs}                                                          % proof trees
\newenvironment{bprooftree}                                                      % boxed proof tree
  {\leavevmode\hbox\bgroup}
  {\DisplayProof\egroup}


% ---------------------------------------------------------------------------- %
% Color and presets
% ---------------------------------------------------------------------------- %

\usepackage[table,xcdraw]{xcolor}                                                % xcolor package with support for tables
\usepackage{colortbl}                                                            % color presets working together with xcolor


% ---------------------------------------------------------------------------- %
% Tikz
% ---------------------------------------------------------------------------- %
\usepackage{tikz}                                                                % import basepackage

% Graphics
\usetikzlibrary{calc}                                                            % Coordinate calcuations
\usetikzlibrary{positioning}                                                     % Relative positioning
\usetikzlibrary{shapes}                                                          % Basic shapes to draw with

% Graphs
\usetikzlibrary{automata}

% Plots
\usepackage{graphicx}                                                            % import basepackage for graphs
\usepackage{pgfplots}                                                            % import pgfplots
\usepgfplotslibrary{fillbetween}                                                 % add fillBetween command
\pgfplotsset{compat=1.15}


% ---------------------------------------------------------------------------- %
% Code: inline
% ---------------------------------------------------------------------------- %
\newcommand{\code}[1]{{\sf #1}}                                                  % \code{X} writes X in a code-appropriate font


% ---------------------------------------------------------------------------- %
% Code: lstlisting
% ---------------------------------------------------------------------------- %
\usepackage{listings}

% General settings
\lstset{
  mathescape=true,                                                               % escape to LaTeX math with $
  escapeinside={*@}{@*},                                                         % if you want to fully escape to LaTeX
  literate={æ}{{\ae}}1{ø}{{\oe}}1{å}{{\aa}}1                                     % allow æ, ø and å in code
           {Æ}{{\AE}}1{Ø}{{\O}}1{Å}{{\AA}}1,                                     %     (this change was taken from the preamble of the MatFysTutor LaTeX Guide)
}

% Formatting settings
\lstset{
  % Formatting inside
  stepnumber=1,                                                                  % step between to line-numbers. 1 = each line is numbered
  numbers=left,                                                                  % numbering: none, left, right
  numbersep=5pt,                                                                 % distance between linenumbers and code
  xleftmargin=\parindent,                                                        % indent linenumbers to not be outside of margin
  xrightmargin=\parindent,                                                       % make the indent symmetrix
  columns=[c]fixed,                                                              % makes it monospaced
  % Formatting of border and caption
  captionpos=b                                                                   % caption at the bottom
}
\DeclareCaptionFormat{listing}{%
  \makebox[2.1cm][l]{\qquad#1#2}%
  \parbox[t]{\dimexpr \captionwidth-2.1cm}{#3}%
}
\captionsetup[lstlisting]{format=listing, singlelinecheck=off, labelsep=colon}


% Whitespace settings
\lstset{
  showspaces=false,                                                              % show spaces everywhere - adding particular underscores
  showstringspaces=false,                                                        % underline spaces within strings only.
  showtabs=false,                                                                % show tabs within strings adding particular underscores.
  breaklines=false,                                                              % do not break lines
  breakatwhitespace=false,                                                       % do not break lines within whitespace
  tabsize=4                                                                      % tab width
}

% Colour settings
\definecolor{lstComment}{rgb}{0.45,0.45,0.45}      % code: comments (Grey)
\definecolor{lstKey}{rgb}{0.13,0.21,1}             % code: primary keywords (Blue)
\definecolor{lstKey2}{rgb}{1,0.666667,0.13726}     % code: secondary keywords (Day[9] Orange)
\definecolor{lstString}{rgb}{0.1,0.65,0.1}         % code: strings (Green)
\definecolor{lstBase}{rgb}{0.0,0.0,0.0}            % code: base (Black)

\lstset{
  numberstyle=\color{lstComment},                     % change style of numbering - currently grey.
  basicstyle=\ttfamily \color{lstBase},  % set basic color
  commentstyle=\color{lstComment},                    % set color of comments
  keywordstyle=[1]\color{lstKey},                     % set color of primary keywords
  keywordstyle=[2]\color{lstKey2},                    % set color of secondary keywords
  stringstyle=\color{lstString},                      % set color of strings
}

% ---------------------------------------------------------------------------- %
% Code: lstlisting languages
% ---------------------------------------------------------------------------- %
\lstdefinelanguage{pseudocode}{
  keywords=[1]{
    break,
    by,
    do,
    downto,
    else,
    error,
    for,
    if,
    let,
    repeat,
    return,
    then,
    to,
    until,
    while,    
  },
  keywords=[2]{
    and,
    or,
    NIL,
  }
  sensitive=false,                                 % keywords are not case-sensitive
  morecomment=[l]{//},                             % l is for line comment
  morecomment=[s]{/*}{*/},                         % s is for start and end delimiter
  morestring=[b]"                                  % strings are enclosed in double quotes
}

% Standard language
\lstset{
  language=pseudocode,
}

% ---------------------------------------------------------------------------- %
% Variables for internal customization (multiple languages)
% ---------------------------------------------------------------------------- %
\usepackage{pgfkeys}

\newcommand{\setvalue}[1]{\pgfkeys{/variables/#1}}
\newcommand{\getvalue}[1]{\pgfkeysvalueof{/variables/#1}}
\newcommand{\declare}[1]{%
  \pgfkeys{
    /variables/#1.is family,
    /variables/#1.unknown/.style = {\pgfkeyscurrentpath/\pgfkeyscurrentname/.initial = ##1}
  }%
}

\declare{}

% import standard variables
% LOAD THIS AFTER preamble_base_p1.tex !

% ---------------------------------------------------------------------------- %
% lstlisting
% ---------------------------------------------------------------------------- %
\setvalue{lst_name = Code}
\setvalue{lsts_name = List of code}

% ---------------------------------------------------------------------------- %
% Proofs environments
% ---------------------------------------------------------------------------- %
\setvalue{thm_name     = Theorem}
\setvalue{lem_name     = Lemma}
\setvalue{prop_name    = Proposition}
\setvalue{cor_name     = Corollary}
\setvalue{def_name     = Definition}
\setvalue{conj_name    = Conjecture}
\setvalue{rem_name     = Remark}

\setvalue{obs_name     = Observation}
\setvalue{hyp_name     = Hypothesis}
\setvalue{example_name = Example}

\setvalue{proof_name   = Proof}




% FROM HERE LOAD preamble_base_p2.tex with any custom settings in between!

