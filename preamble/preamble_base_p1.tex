% ---------------------------------------------------------------------------- %
% Paper size and encoding
% ---------------------------------------------------------------------------- %

\usepackage[utf8]{inputenc}                                                      % encoding: utf-8 (nordic letters)
\usepackage[T1]{fontenc}                                                         % use 8-bit encoded fonts

% Uncomment to add more whitespace between paragraphs rather than indent
%\usepackage[parfill]{parskip}

\usepackage{csquotes}

% ---------------------------------------------------------------------------- %
% Colour
% ---------------------------------------------------------------------------- %
\usepackage{xcolor}
\usepackage{colortbl}                                                            % color presets working together with xcolor

\definecolor{cGray}{rgb}{0.45,0.45,0.45}
\definecolor{cRed}{rgb}{0.545,0.137,0}
\definecolor{cBlue1}{rgb}{0.13,0.21,1}
\definecolor{cBlue2}{rgb}{0,0.408,0.545}
\definecolor{cCyan}{rgb}{0,0.545,0.545}
\definecolor{cOrange1}{rgb}{1,0.666667,0.13726} % Day[9]tv orange
\definecolor{cOrange2}{rgb}{1.0,0.502,0}
\definecolor{cGreen}{rgb}{0.1,0.65,0.1}
\definecolor{cBlack}{rgb}{0.0,0.0,0.0}

% ---------------------------------------------------------------------------- %
% Floats: Tables and figures
% ---------------------------------------------------------------------------- % 
\usepackage{float, tabularx, booktabs}                                           % various basic stuff for tables and more
\usepackage{caption, subcaption}                                                 % captions and sub figures


\usepackage{wrapfig}                                                             % figures wrapped by text

\usepackage{rotating}                                                            % rotate content (i.e. figures sideways)

% ---------------------------------------------------------------------------- %
% Mathematics
% ---------------------------------------------------------------------------- %
% various basic stuff
\usepackage{mathtools, amsmath, amssymb, bbm}
\usepackage{stmaryrd}                                                            % even more symbols

% Pull requests with more macros are very welcome!

% ---------------------------------------------------------------------------- %
% Linear Algebra
% ---------------------------------------------------------------------------- %
\newcommand{\allones}{\ensuremath{\mathbf{1}}}                                  % \allones  = The all ones vector
\newcommand{\allzeros}{\ensuremath{\mathbf{0}}}                                 % \allzeros = The all zeros vector

% ---------------------------------------------------------------------------- %
% Calligraphic symbols ('cal' prefix)
% ---------------------------------------------------------------------------- %
\newcommand{\calF}{\ensuremath{\mathcal{F}}}
\newcommand{\calG}{\ensuremath{\mathcal{G}}}
\newcommand{\calH}{\ensuremath{\mathcal{H}}}
\newcommand{\calO}{\ensuremath{\mathcal{O}}}
\newcommand{\calS}{\ensuremath{\mathcal{S}}}
\newcommand{\calV}{\ensuremath{\mathcal{V}}}
\newcommand{\calW}{\ensuremath{\mathcal{W}}}

% ---------------------------------------------------------------------------- %
% Algebra
% ---------------------------------------------------------------------------- %
\renewcommand{\d}{\, \mathrm{d}}                                                 % \d = differential d with a bit of spacing
\newcommand{\e}{\ensuremath{\mathrm{e}}}                                         % \e = eulers number
\newcommand{\R}{\ensuremath{\mathbb{R}}}                                         % \R = Real numbers
\newcommand{\N}{\ensuremath{\mathbb{N}}}                                         % \N = Natural numbers
\newcommand{\Z}{\ensuremath{\mathbb{Z}}}                                         % \Z = Integers
\newcommand{\C}{\ensuremath{\mathbb{C}}}                                         % \C = Complex numbers
\newcommand{\Q}{\ensuremath{\mathbb{Q}}}                                         % \Q = Rational numbers
\newcommand{\F}{\ensuremath{\mathbb{F}}}                                         % \F = Field
\newcommand{\K}{\ensuremath{\mathbb{K}}}                                         % \K = Field \R and \C
\renewcommand{\S}{\ensuremath{\mathbb{S}}}                                       % \S = Group of permutations

% ---------------------------------------------------------------------------- %
% Complexity Theory ('c' prefix)
% ---------------------------------------------------------------------------- %
% Complexity classes
\newcommand{\cDSPACE}{\text{DSPACE}}                                             % \cDSPACE   = DSPACE
\newcommand{\cDTIME}{\text{DTIME}}                                               % \cDTIME    = DTIME
\newcommand{\cNSPACE}{\text{NSPACE}}                                             % \cNSPACE   = NSPACE
\newcommand{\cNTIME}{\text{NTIME}}                                               % \cNTIME    = NTIME

\newcommand{\cCLASS}[1]{\ensuremath{\mathrm{#1}}}

\newcommand{\cL}{\cCLASS{L}}                                                     % \cL        = Deterministic Logarithmic Space
\newcommand{\cNL}{\cCLASS{NL}}                                                   % \cNL       = Nondeterministic Logarithmic Space
\newcommand{\cP}{\cCLASS{P}}                                                     % \cP        = Deterministic Polynomial Time
\newcommand{\cNP}{\cCLASS{NP}}                                                   % \cNP       = Nondeterministic Polynomial Time
\newcommand{\ccoNP}{\cCLASS{coNP}}                                               % \ccoNP     = Conondeterministic Polynomial Time
\newcommand{\cPSPACE}{\cCLASS{PSPACE}}                                           % \cPSPACE   = Deterministic Polynomial Space
\newcommand{\cEXP}{\cCLASS{EXP}}                                                 % \cEXP      = Deterministic Polynomial Time
\newcommand{\cNEXP}{\cCLASS{NEXP}}                                               % \cNEXP     = Nondeterministic Polynomial Time
\newcommand{\cEXPSPACE}{\cCLASS{EXPSPACE}}                                       % \cEXPSPACE = Deterministic Polynomial Space

\newcommand{\cSqrtSum}{\cCLASS{SqrtSum}}                                          % \SqrtSum   = Square Root sum complexity class
\newcommand{\cETR}{\cCLASS{\exists\R}}                                           % \cETR      = Existential Theory of the Reals

\newcommand{\cBPP}{\cCLASS{BPP}}                                                 % \cBPP      = Polynomial time randomness with 2-sided error
\newcommand{\cFPT}{\cCLASS{FPT}}                                                 % \cFPT      = Fixed parameter tractable

% Complete problems
\newcommand{\cSTCON}{\textsc{STCON}}                                             % \cSTCON    = ST Connectivity
\newcommand{\cSAT}{\textsc{SAT}}                                                 % \cSAT      = Satisfiability
\newcommand{\cILP}{\textsc{ILP}}                                                 % \cILP      = Integer Linear Programming
\newcommand{\cOVP}{\textsc{OVP}}                                                 % \cOVP      = Orthogonal Vectors

% Hypothesis
\newcommand{\cETH}{\textsc{eth}}                                                 % \cETH      = Exponential Time Hypothesis
\newcommand{\cSETH}{\textsc{Seth}}                                               % \cSETH     = Strong ETH

% ---------------------------------------------------------------------------- %
% Simple macros
% ---------------------------------------------------------------------------- %
\newcommand{\Id}{\text{Id}}                                                      % t\Id = Identity function

\newcommand{\Det}[1]{\text{Det}\left( #1 \right)}                                % \Det{arg}             Det(arg)
\newcommand{\Span}[1]{\text{Span}\left( #1 \right)}                              % \Span{arg}            Span(arg)
\newcommand{\sgn}[1]{\text{sgn} \left( #1 \right)}                               % \sgn{arg}             sgn(arg)
\newcommand{\adj}[1]{\text{adj} \left( #1 \right)}                               % \adj{arg}             adj(arg)
\newcommand{\ord}[1]{\text{ord} \left( #1 \right)}                               % \ord{arg}             ord(arg)

\newcommand{\tuple}[1]{\left\langle #1 \right\rangle}                            % \tuple{arg}           <arg>
\newcommand{\abs}[1]{\left\lvert #1 \right\rvert}                                % \abs{arg}             absolute/modulo of value
\newcommand{\norm}[1]{\left\lVert #1 \right\rVert}                               % \norm{arg}            norm of a value
\newcommand{\ceil}[1]{\left\lceil #1 \right\rceil}                               % \ceil{arg}            ceiling of a value
\newcommand{\floor}[1]{\left\lfloor #1 \right\rfloor}                            % \floor{arg}           floor of a value
\newcommand{\inprod}[2]{\tuple{#1, #2}}                                          % \inprod{v}{w}         inner product
\newcommand{\powgroup}[1]{\tuple{#1}}                                            % \powgroup{arg}        image of f={g^n | n in Z}

% ---------------------------------------------------------------------------- %
% Advanced macros:
% ---------------------------------------------------------------------------- %
\usepackage{xparse}                                                              % Scanning arguments
\usepackage{xifthen}                                                             % Conditionals
\usepackage{xstring}                                                             % String functions
\usepackage{calc}                                                                % Calculations

\newcounter{i}

\DeclareDocumentCommand \set { m g g }{                                          % \set{X}{C}{|}
     \left\lbrace                                                                % {X | C}
         #1 \IfValueT {#2} { \ \IfValueTF{#3}{#3}{|}\  #2 }
     \right\rbrace
}

\DeclareDocumentCommand \seq { g g g g } {                                       % \seq{x}{i}{j}{s}
    \setcounter{i}{0}                                                            % x_i, x_i+s, ... x_j
    \IfValueT {#2} { \addtocounter{i}{#2} }
    \IfValueTF {#1}
        {#1}
        {x}
    _{ \arabic{i} },
    \IfValueTF {#4} 
        {\addtocounter{i}{#4}}
        {\addtocounter{i}{1}}
    \IfValueTF {#1} 
        {#1}
        {x} 
    _{ \arabic{i} },
    \dots
    \IfValueTF {#3}
        { , #1_{#3} }
        {}
}

\DeclareDocumentCommand \eqclass { g g }{                                        % \eqclass{v}{V}    Equivalent Class
    \left[                                                                       % This can also be used for coordinate vectors
        \IfValueTF{#1}
            {#1}
            {\dot}
    \right]
    \IfValueTF{#2}
            {_{#2}}
            {}
}

\DeclareDocumentCommand \ero { g g } {                                           % \ero {x, y}
    \begin{array}{c}                                                             %    x
        \IfValueTF{#1}                                                           %    ~
            {_{#1}}                                                              %    y
            {\phantom{\sim}}
    \\
        \sim
    \\
        \IfValueTF{#2}
            {^{#2}}
            {\phantom{\sim}}
    \end{array}
}

\DeclareDocumentCommand \matrep { g g g } {                                      % \matrep{W}{L}{V}    Matrixrepresentation
    {_{                                                                          % W[L]V
        \IfValueTF {#1}                                                          %   No arguments for W or V results in standard basis
            {#1}                                                                 %   No arguments for L results in the coordinate transformation
            {\epsilon}
    }}
    \left[
        \IfValueTF {#2}
            {#2}
            {\square}
    \right] {_{
        \IfValueTF {#3}
            {#3}
            {\epsilon}
    }}
}

\newcommand{\IndexedFunc}[3]{{#1}_{#2} \left( #3 \right)}

\DeclareDocumentCommand \Geo { g g }{                                            % \Geo{v}{V}    Geogrebic Multiplicity
    \IndexedFunc
        {\text{Geo}}
        {\IfValueTF{#1}
                {#1}
                {L}}
        {\IfValueTF{#2}
            {#2}
            {\lambda}}
}

\DeclareDocumentCommand \Alg { g g }{                                            % \Alg            Algebraic Multiplicity
    \IndexedFunc
        {\text{Alg}}
        {\IfValueTF{#1}
                {#1}
                {L}}
        {\IfValueTF{#2}
            {#2}
            {\lambda}}
}

\DeclareDocumentCommand \series { g g g g }{                                     % \series{a}{b}{c}{d}    \{ a_b \} _{b = c} ^d
    \set{\IfValueTF{#1}                                                          %    An infinite series with elements a, indexed by b
      {#1}                                                                       %    starting from c ending at d
      {a}
     _{\IfValueTF{#2}
      {#2}
      {n}}      
      }
      _{\IfValueTF{#2}
        {#2}
        {n}
       =
       \IfValueTF{#3}
        {#3}
        {1}
      }
      ^{\IfValueTF{#4}
        {#4}
        {\infty}
      }
}

\DeclareDocumentCommand \pseries { g g g g }{                                    % \pseries{a}{b}{c}{d}   \{ x^k \} _{k = c} ^d
    \set{\IfValueTF{#1}                                                          %    An infinite series of points x, indexed by k
      {#1}                                                                       %    starting from c ending at d
      {x}
     ^{\IfValueTF{#2}
      {#2}
      {k}}      
      }
      _{\IfValueTF{#2}
        {#2}
        {k}
       =
       \IfValueTF{#3}
        {#3}
        {1}
      }
      ^{\IfValueTF{#4}
        {#4}
        {\infty}
      }
}

\DeclareDocumentCommand \infseq { g g g g }{                                     % \infseq{a}{b}{c}{d}   \sum_{ b = c }^d a_b
  \sum                                                                           %    An infinite sequence with elements a, indexed by b
    _{                                                                           %    starting from c ending at d
        \IfValueTF{#2}
          {#2}
          {n}
         =
         \IfValueTF{#3}
          {#3}
          {1}
    }
      ^{\IfValueTF{#4}
        {#4}
        {\infty}
      }
    \IfValueTF{#1}
      {#1}
      {a}
   _{\IfValueTF{#2}
      {#2}
      {n}
  }     
}

\renewcommand{\|}{\scalebox{1.3}{|}\ }                                           % Larger vertical bar for divisors
\renewcommand{\div}[1]{\text{div}\left( #1 \right)}                              % div(arg)


% adds vertical lines to matrices
\makeatletter
\renewcommand*\env@matrix[1][*\c@MaxMatrixCols c]{
  \hskip -\arraycolsep
  \let\@ifnextchar\new@ifnextchar
  \array{#1}}
\makeatother


% ---------------------------------------------------------------------------- %
% Logic and proofs
% ---------------------------------------------------------------------------- %
% Proofs
\usepackage{amsthm}                                                              % Theorem package
\theoremstyle{definition}                                                        % Style: plain, definition, remark

% Uncomment if you want to have the number first
%\swapnumbers

% Logic packages
%\usepackage{lplfitch}                                                            % fitch style proofs
\usepackage{logicproof}                                                         % fitch style proofs

\usepackage{bussproofs}                                                          % proof trees
\newenvironment{bprooftree}                                                      % boxed proof tree
  {\leavevmode\hbox\bgroup}
  {\DisplayProof\egroup}


% ---------------------------------------------------------------------------- %
% Tikz
% ---------------------------------------------------------------------------- %
\usepackage{tikz}                                                                % import basepackage

% Graphics
\usetikzlibrary{calc}                                                            % Coordinate calcuations
\usetikzlibrary{positioning}                                                     % Relative positioning
\usetikzlibrary{shapes}                                                          % Basic shapes to draw with
\usetikzlibrary{arrows}                                                          % Arrow customization
\usetikzlibrary{patterns}                                                        % Patterns in drawings

% Graphs
\usetikzlibrary{automata}

% Plots
\usepackage{graphicx}                                                            % import basepackage for graphs
\usepackage{pgfplots}                                                            % import pgfplots
\usepgfplotslibrary{fillbetween}                                                 % add fillBetween command
\pgfplotsset{compat=1.15}

% qtree
\usepackage[noload]{qtree}


% ---------------------------------------------------------------------------- %
% Code: lstlisting
% ---------------------------------------------------------------------------- %
\usepackage{listings}

% Centered (float-like) listings (code from: https://tex.stackexchange.com/a/245750)
\ExplSyntaxOn
\tl_new:N \l_listings_boxed_options_tl
\keys_define:nn { listings/boxed }
 {
  caption .tl_set:N = \l_listings_boxed_caption_tl,
  shortcaption .tl_set:N = \l_listings_boxed_shortcaption_tl,
  label .tl_set:N = \l_listings_boxed_label_tl,
  unknown .code:n =
          \tl_put_right:NV \l_listings_boxed_options_tl \l_keys_key_tl
          \tl_put_right:Nn \l_listings_boxed_options_tl { = #1 , },
 }
\box_new:N \l_listings_boxed_box

\lstnewenvironment{blstlisting}[1][]
 {
  \keys_set:nn { listings/boxed } { #1 }
  \exp_args:NV \lstset \l_listings_boxed_options_tl
  \hbox_set:Nw \l_listings_boxed_box
 }
 {
  \hbox_set_end:
  \cs_set_eq:cc {c@figure} {c@lstlisting}
  \tl_set_eq:NN \figurename \lstlistingname
  \tl_if_empty:NF \l_listings_boxed_caption_tl
   {pp
    \tl_if_empty:NTF \l_listings_boxed_shortcaption_tl
     {
      \captionof{figure}{\l_listings_boxed_caption_tl}
     }
     {
      \captionof{figure}[\l_listings_boxed_shortcaption_tl]{\l_listings_boxed_caption_tl}
     }
    \tl_if_empty:NF \l_listings_boxed_label_tl { \label{\l_listings_boxed_label_tl} }
   }
  \leavevmode\box_use:N \l_listings_boxed_box
 }
\ExplSyntaxOff

% Float listings
\newfloat{lstfloat}{htbp}{lop}

% General settings
\lstset{
  mathescape=true,                                                               % escape to LaTeX math with $
  escapeinside={*@}{@*},                                                         % if you want to fully escape to LaTeX
  literate={æ}{{\ae}}1{ø}{{\oe}}1{å}{{\aa}}1                                     % allow æ, ø and å in code
           {Æ}{{\AE}}1{Ø}{{\O}}1{Å}{{\AA}}1,                                     %     (this change was taken from the preamble of the MatFysTutor LaTeX Guide)
}

% Formatting settings
\lstset{
  % Formatting inside
  stepnumber=1,                                                                  % step between to line-numbers. 1 = each line is numbered
  numbers=left,                                                                  % numbering: none, left, right
  numbersep=5pt,                                                                 % distance between linenumbers and code
  xleftmargin=\parindent,                                                        % indent linenumbers to not be outside of margin
  xrightmargin=\parindent,                                                       % make the indent symmetrix
  columns=[c]fixed,                                                              % makes it monospaced
  % Formatting of border and caption
  captionpos=b                                                                   % caption at the bottom
}
\DeclareCaptionFormat{listing}{%
  \makebox[2.1cm][l]{\qquad#1#2}%
  \parbox[t]{\dimexpr \captionwidth-2.1cm}{#3}%
}
\captionsetup[lstlisting]{format=listing, singlelinecheck=off, labelsep=colon}


% Whitespace settings
\lstset{
  showspaces=false,                                                              % show spaces everywhere - adding particular underscores
  showstringspaces=false,                                                        % underline spaces within strings only.
  showtabs=false,                                                                % show tabs within strings adding particular underscores.
  breaklines=false,                                                              % do not break lines
  breakatwhitespace=false,                                                       % do not break lines within whitespace
  tabsize=4                                                                      % tab width
}

% Colour settings
\lstset{
  numberstyle=\color{cGray},
  basicstyle=\small\upshape\ttfamily,
  commentstyle=\color{cGray},
  keywordstyle=[1]\color{cBlue1},
  stringstyle=\color{cGreen},
}

% ---------------------------------------------------------------------------- %
% Code: lstlisting languages
% ---------------------------------------------------------------------------- %
\lstdefinelanguage{none}{
  morecomment=[l]{//},
  morecomment=[s]{/*}{*/},
  morestring=[b]"
}

\lstdefinelanguage{pseudocode}{
  keywords=[1]{
    break, by, continue, case, do, downto, else, error, for, if, let, of,
    repeat, return, then, to, until, while, NIL,
  },
  sensitive=false,
  morecomment=[l]{//},
  morecomment=[s]{/*}{*/},
  morestring=[b]"
}

% from: https://gist.github.com/nikolajquorning/92bbbeef32e1dd80105c9bf2daceb89a

\lstdefinelanguage{sml}{
  keywords=[1]{
    EQUAL, GREATER, LESS, NONE, SOME, abstraction, abstype, and, andalso, array,
    as, before, bool, case, char, datatype, do, else, end, eqtype, exception,
    exn, false, fn, fun, functor, handle, if, in, include, infix, infixr, int,
    let, list, local, nil, nonfix, not, o, of, op, open, option, orelse,
    overload, print, raise, real, rec, ref, sharing, sig, signature, string,
    struct, structure, substring, then, true, type, unit, val, vector, where,
    while, with, withtype, word
  },
  sensitive=false,
  morecomment=[s]{(*}{*)},
  morestring=[b]"
}

% From: https://hal.archives-ouvertes.fr/file/index/docid/594606/filename/lstcoq.sty

\lstdefinelanguage{Coq}{ 
  texcl=false,
  sensitive=true,
  % Vernacular commands
  morekeywords=[1]{
    Section, Module, End, Require, Import, Export, Variable, Variables,
    Parameter, Parameters, Axiom, Hypothesis, Hypotheses, Notation, Local,
    Tactic, Reserved, Scope, Open, Close, Bind, Delimit, Definition, Let, Ltac,
    Fixpoint, CoFixpoint, Add, Morphism, Relation, Implicit, Arguments, Unset,
    Contextual, Strict, Prenex, Implicits, Inductive, CoInductive, Record,
    Structure, Canonical, Coercion, Context, Class, Global, Instance, Program,
    Infix, Theorem, Lemma, Corollary, Proposition, Fact, Remark, Example, Proof,
    Goal, Save, Qed, Defined, Hint, Resolve, Rewrite, View, Search, Show, Print,
    Printing, All, Eval, Check, Projections, inside, outside, Def
  },
  % Gallina
  morekeywords=[2]{
    forall, exists, exists2, fun, fix, cofix, struct, match, with, end, as, in,
    return, let, if, is, then, else, for, of, nosimpl, when},
  % Sorts
  morekeywords=[3]{Type, Prop, Set, true, false, option},
  % Various tactics, some are std Coq subsumed by ssr, for the manual purpose
  morekeywords=[4]{
    pose, set, move, case, elim, apply, clear, hnf, intro, intros, generalize,
    rename, pattern, after, destruct, induction, using, refine, inversion,
    injection, rewrite, congr, unlock, compute, ring, field, fourier, replace,
    fold, unfold, change, cutrewrite, simpl, have, suff, wlog, suffices,
    without, loss, nat_norm, assert, cut, trivial, revert, bool_congr,
    nat_congr, symmetry, transitivity, auto, split, left, right, autorewrite},
  % Terminators
  morekeywords=[5]{
    by, done, exact, reflexivity, tauto, romega, omega, assumption, solve,
    contradiction, discriminate},
  % Control
  morekeywords=[6]{do, last, first, try, idtac, repeat},
  % Comments delimiters, we do turn this off for the manual
  morecomment=[s]{(*}{*)},
  % String delimiters
  morestring=[b]",
  morestring=[d]’,
  % Size of tabulations
  tabsize=3,
  % Enables ASCII chars 128 to 255
  extendedchars=false,
  % Style for (listings') identifiers
  identifierstyle=\ttfamily\color{cBlack},
  % Style for declaration keywords
  keywordstyle=[1]\color{cBlack}\textit,
  % Style for gallina keywords
  keywordstyle=[2]\color{cBlue1},
  % Style for sorts keywords
  keywordstyle=[3]\color{cBlue2},
  % Style for tactics keywords
  keywordstyle=[4]\color{cRed},
  % Style for terminators keywords
  keywordstyle=[5]\color{cOrange2},
  % Style for iterators
  keywordstyle=[6]\color{cRed},
  % Style for strings
  stringstyle=\ttfamily,
  % Style for comments
  commentstyle={\ttfamily\color{cGreen}},
  literate=
    {\\forall}{{\color{dkgreen}{$\forall\;$}}}1
    {\\exists}{{$\exists\;$}}1
    {<-}{{$\leftarrow\;$}}1
    {=>}{{$\Rightarrow\;$}}1
    {==}{{\textsc{==}\;}}1
    {==>}{{\textsc{==>}\;}}1
    % {:>}{{\code{:>}\;}}1
    {->}{{$\rightarrow\;$}}1
    {<->}{{$\leftrightarrow\;$}}1
    {<==}{{$\leq\;$}}1
    {\#}{{$^\star$}}1 
    {\\o}{{$\circ\;$}}1 
    {\@}{{$\cdot$}}1 
    {\/\\}{{$\wedge\;$}}1
    {\\\/}{{$\vee\;$}}1
    {++}{{\textsc{++}}}1
    {~}{{\ }}1
    {\@\@}{{$@$}}1
    {\\mapsto}{{$\mapsto\;$}}1
    {\\hline}{{\rule{\linewidth}{0.5pt}}}1
  }[keywords,comments,strings]


% Default language
\lstset{
  language=pseudocode,
}


% ---------------------------------------------------------------------------- %
% Author tools: TODO
% ---------------------------------------------------------------------------- %
\usepackage[obeyFinal]{todonotes}

% Citation TODO
\DeclareDocumentCommand \todocite { g }{
  [
    {\bf \color{cOrange2} \IfValueTF{#1}{#1}{??}}
  ]
}

% ---------------------------------------------------------------------------- %
% Hypermedia
% ---------------------------------------------------------------------------- %

\usepackage{url}                                                                 % \url{link}
\usepackage[bookmarks=true, bookmarksopen=true, implicit=false]{hyperref}        % \href{link}{replacing text}

\newcommand*{\http}[1]{\href{http://#1}{#1}}                                     % macro for http links: \http{www.matfystutor.dk}
\newcommand*{\mailto}[1]{\href{mailto:#1}{\nolinkurl{#1}}}                       % macro for mails: \mailto{email@email.com}


% ---------------------------------------------------------------------------- %
% Variables for internal customization (multiple languages)
% ---------------------------------------------------------------------------- %
\usepackage{pgfkeys}

\newcommand{\setvalue}[1]{\pgfkeys{/variables/#1}}
\newcommand{\getvalue}[1]{\pgfkeysvalueof{/variables/#1}}
\newcommand{\declare}[1]{%
  \pgfkeys{
    /variables/#1.is family,
    /variables/#1.unknown/.style = {\pgfkeyscurrentpath/\pgfkeyscurrentname/.initial = ##1}
  }%
}

\declare{}

% import standard variables
% LOAD THIS AFTER preamble_base_p1.tex !

% ---------------------------------------------------------------------------- %
% lstlisting
% ---------------------------------------------------------------------------- %
\setvalue{lst_name = Code}
\setvalue{lsts_name = List of code}

% ---------------------------------------------------------------------------- %
% Proofs environments
% ---------------------------------------------------------------------------- %
\setvalue{thm_name     = Theorem}
\setvalue{lem_name     = Lemma}
\setvalue{prop_name    = Proposition}
\setvalue{cor_name     = Corollary}
\setvalue{def_name     = Definition}
\setvalue{conj_name    = Conjecture}
\setvalue{rem_name     = Remark}

\setvalue{obs_name     = Observation}
\setvalue{hyp_name     = Hypothesis}
\setvalue{example_name = Example}

\setvalue{proof_name   = Proof}



% FROM HERE LOAD preamble_base_p2.tex with any custom settings in between!

